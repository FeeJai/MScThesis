\chapter*{Abstract}

\begin{center}
\begin{minipage}[t]{.8\textwidth}
\setlength{\parskip}{.5\baselineskip}


Gathering data from audio recordings currently relies primarily on hand-crafted and fine-tuned signal processing methods.
A common example is the Mel Frequency Cepstral Coefficients \cite{1168654}, a voice pitch independent representation of the time variant properties in the speech spectrum \cite{Hanson1996}.

A lot of research has been done in the past on how to combine these feature extraction methods with common machine learning algorithms for the purpose of speech recognition.
However, literature on the detection, recognition, and classification of other sounds like urban noise is a lot more rare.

The purpose of this thesis is to explore how these methods can be transferred onto sounds others than speech and how this can be used to improve traffic safety.


\end{minipage}
\end{center}
